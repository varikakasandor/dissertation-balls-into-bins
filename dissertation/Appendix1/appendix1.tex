%!TEX root = ../thesis.tex

\chapter{Notes on Related Work and Background Reading}\label{related-work} 

Following is a summary of my background reading and my observations:


\begin{itemize}
    \item 
    I wanted to understand why there are so many different protocols, see how they are interconnected, and explore the main directions of current research. These allowed me to choose in a principled way which protocols to focus on. In particular, \TwoThinning, the main protocol of my study, has been analysed as a resource efficient protocol in-between \OneChoice and \TwoChoice. (We can see a similar phenomenon with the so-called \textsc{($1+\beta$)-process}~\cite{peres2015oneplusbeta}).
    \item
    I found that most of the papers in this area approach the topic from a theoretical viewpoint, and there is a big gap between these results and the practical applications. Even though my approach is also a theoretical one, I found it important to gain practical motivation, and therefore I actively searched for papers bridging this gap (e.g.~\cite{wang2017twochoicerouting}).
    \item
    As preparation for formulating my own lemmas and conjectures for the dissertation, I wanted to gain an insight into how lower and upper bounds can be derived theoretically on the maximum load of various protocols. Therefore I studied several proofs, e.g.\ that \TwoChoice achieves a maximum load of $\frac{\ln\ln n}{\ln 2} + O(1)$ after $n$ balls~\cite{azar1999twochoice}. These proofs did not just provide insights into why or why not different protocols work well, but the proofs also provide intuition for improving protocols, or what is even more important for my project, creating good strategies. For example, the so-called \Threshold strategy for \TwoThinning is based on the observation that rejected balls form a \OneChoice process, for which tight bounds are available~\cite{feldheim2021thinning}.
    \item
    Perhaps surprisingly, the more general case where $m\geq n$ is much more challenging than the $m=n$ case (i.e.\ where the number of balls equal the number of bins)~\cite{berenbrink2006heavilyloaded}. In real-world applications this assumption may usually not hold, only if the number of jobs is somehow controlled. My approaches outlined in Chapter~\ref{implementation} are not limited by this constraint.
    \iffalse % I discussed it all already in the Introduction, I think
    \item
    Almost exclusively all of the proofs are asymptotic -- they only hold for very large $n$, mostly due to approximations, such as Stirling's formula~\cite{feldheim2021thinning}. This is very restrictive for real-world applications, as discussed in Chapter~\ref{introduction}. Similarly, (and sometimes also as a consequence of the above), the bounds hold only up to a constant/logarithmic factor, i.e.\ they are fixed only up to the $\Theta$ notation. In particular, there exist strategies that have been shown to be optimal for \TwoThinning, such as the ``$\ell$-threshold strategy'' for $m=n$, but this is also not necessarily optimal for practically realistic values of $n$ and $m$! \NOTE{A}{Better phrasing. An interesting idea that I will get back to in Chapter~\ref{evaluation} is trying to find out or bound the constant factors hidden behind the $\Theta$ notation, which would be very useful for comparisons.}
    \NOTE{D}{This sounds interesting}
    \fi
\end{itemize}