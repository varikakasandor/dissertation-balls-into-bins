\begin{proforma}      


\begin{table}[h]
\begin{tabular}{ll}
Candidate Number:  & 2433E \\
College: & Churchill College \\
Project Title:    &  Reinforcement Learning meets Balls-into-Bins  \\
Examination:  & Computer Science Tripos - Part II, May 2022   \\
Word Count:  & \quickwordcount{thesis} \protect\footnotemark[1]\\
Lines of Code: & 8962 \protect\footnotemark[2] \\
Project Originator: & Dr Thomas Sauerwald, Dimitris Los \\
Supervisors: & Dr Thomas Sauerwald, Dimitris Los 
\end{tabular}
\end{table}
\NOTE{A}{Replace 0 with actual numbers.}

%\footnotetext[1]{Computed using texcount: \url{http://app.uio.no/ifi/texcount}}

\footnotetext[1]{Computed using texcount: \url{http://app.uio.no/ifi/texcount}}

\footnotetext[2]{Computed using cloc: \url{https://github.com/AlDanial/cloc}}

\section*{Original Aims of the Project}

The original aim of the project was to explore the applicability of reinforcement learning, dynamic programming and other methods to optimising load balancing protocols. In particular, to find suitable decision strategies for parametric balls-into-bins protocols, which are abstractions of randomised load balancing.

\section*{Work Completed}

I have completed the success criteria, investigating reinforcement learning approaches to different balls-into-bins protocols, contrasting them with dynamic programming and heuristic algorithms, and analysing them all in a modular, easy-to-use environment. As extensions, I formulated conjectures and proved several lemmas about the behaviour of different strategies.

\section*{Special Difficulties}

None.

\end{proforma}
