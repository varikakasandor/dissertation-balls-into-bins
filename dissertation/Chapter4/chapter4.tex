%!TEX root = ../thesis.tex
\chapter{Evaluation}\label{evaluation}


\ifpdf
    \graphicspath{{Chapter3/Figs/Raster/}{Chapter3/Figs/PDF/}{Chapter3/Figs/}}
\else
    \graphicspath{{Chapter3/Figs/Vector/}{Chapter3/Figs/}}
\fi


\NOTE{A}{I am really not sure about the best structure for this chapter.}

\section{General Notes}

In this chapter I provide quantitative and qualitative analysis and comparison of the settings presented in previous chapters.


To obtain thorough and statistically significant results, I utilised parallel processing as much as possible for evaluating Deep Q-Learning policies, which are the bottleneck of the evaluation phase. To exploit the parallelism of the GPU even with the inherently sequential MDP, I ran several (usually $64$) independent samples in parallel. Note that this is much simpler for \textsc{Two-Thinning} and \textsc{Graphical Two Choice}, where the number of steps from start to end in the MDP is constant ($m$) in any execution.


Even with this evaluation speedup, the usability of Deep Q-Learning for large values of $n$ and $m$ is still limited due to the training time. The largest range for which I successfully trained a RL algorithm in at most a day is $n=1000$, $m=1500$, but I decided to focus on slightly smaller values for the evaluation as that is more illustrative and complementary to the literature available.



\NOTE{A}{TODO: Note that in this project there is no train or test set, there isn’t any data at all. Nevertheless, it is still important to do evaluation independently of early stopping, due to the Expectation-Maximum Jensen inequality.}


\section{Two-Thinning}


\subsection{Comparison of Strategies}

Now I present a comparison of seven \textsc{Two-Thinning} strategies. I have chosen nine combinations of $n$ (number of bins) and $m$ (number of balls) so that they cover different ranges, and also different balls-bins ratios.


To have acceptable statistical significance but also reasonable runtime, I compared the strategies across $100$ runs. I show the average of scores (maximum load) of the runs and also the $95\%$ confidence interval for the estimated mean score, based on the Central Limit Theorem (CLT). \NOTE{A}{Should I explain more? Any formal statistical reason for choosing the number of runs?}


Note that the Dynamic Programming (DP) Strategy is optimal by definition, but it is too slow for larger values of $n$ and $m$ which I denote by TLE (Time Limit Exceeded). I decided to limit all the algorithms to $12$ hours of training/preparation, declaring anything above that as TLE. Also, even though we have the exact expected maximum load of the DP Strategy, I decided to run it just like the other strategies for fair comparison (e.g. large outliers that the DP Strategy takes into account occur very rarely during execution).


For the Threshold Strategy, I calculated the optimal values by modelling it as a Multi-Armed Bandit problem (covered in II Randomised Algorithms) \cite{katehakis1987multiarmedbandit}, and using an $\epsilon$-greedy strategy with $\epsilon=0.1$ (details omitted).


I note that there is merit in not just doing several runs with the trained trained Deep Q-Learning model, but also retraining it several times (e.g. $20$ runs with each of $5$ independently trained model), because the rewards received during training are also probabilistic. This idea would rather evaluate the robustness of the Deep Q-Learning framework for balls into bins settings, but I decided to stick to a single trained model and essentially evaluate the peak performance. \NOTE{A}{Does it make any sense? This idea came to my mind and I am a bit unsure about it and couldn't find any reference.}. Relatedly, one could consider finding the optimal hyperparameters as a special part of training, but I decided treat it separately, and in particular, not take it into account for the $12$ hour TLE limit.



\begin{table}[h!]
\caption{Average maximum load of \textsc{Two-Thinning} strategies with $95\%$ confidence intervals}
\label{tab:two-thinning-comparison}
\centering
\resizebox{\textwidth}{!}{%
\begin{tabular}{|l|c|c|c|c|c|c|c|c|c|}
\hline
                                & \multicolumn{3}{c|}{$n=5$} & \multicolumn{3}{c|}{$n=20$} & \multicolumn{3}{c|}{$n=50$} \\ \hline
                                & $m=5$ & $m=10$ & $m=15$ & $m=20$ & $m=60$ & $m=400$ & $m=50$ & $m=200$ & $m=2500$ \\ \hline
Always Accept Strategy          & 2.26 $\pm$ 0.06 & 3.84 $\pm$ 0.1 & 7.88 $\pm$ 0.14 & 3.26 $\pm$ 0.1 & 6.68 $\pm$ 0.15 & 28.84 $\pm$ 0.32 & 3.87 $\pm$ 0.1 & 9.12 $\pm$ 0.16 & 66.66 $\pm$ 0.5 \\ \hline
Random Strategy                 & 2.35 $\pm$ 0.07 & 3.72 $\pm$ 0.09 & 7.66 $\pm$ 0.14 & 3.21 $\pm$ 0.1 & 6.67 $\pm$ 0.15 & 28.55 $\pm$ 0.32 & 3.8 $\pm$ 0.1 & 9.02 $\pm$ 0.18 & 66.83 $\pm$ 0.48 \\ \hline
Local Reward Optimiser Strategy & 1.82 $\pm$ 0.05 & 2.97 $\pm$ 0.05 & 6.08 $\pm$ 0.06 & 2.25 $\pm$ 0.06 & 4.75 $\pm$ 0.08 & 22.46 $\pm$ 0.09 & 2.54 $\pm$ 0.07 & 6.37 $\pm$ 0.07 & 53.98 $\pm$ 0.11 \\ \hline
Mean Thinning Strategy          & 1.87 $\pm$ 0.07 & 3.1 $\pm$ 0.07 & 6.17 $\pm$ 0.09 & 2.56 $\pm$ 0.08 & 5.12 $\pm$ 0.1 & 22.52 $\pm$ 0.12 & 3.06 $\pm$ 0.08 & 6.79 $\pm$ 0.11 & 53.53 $\pm$ 0.15 \\ \hline
DP Strategy                     & 1.85 $\pm$ 0.06 & 2.98 $\pm$ 0.07 & 6.06 $\pm$ 0.1 & 2.33 $\pm$ 0.1 & 4.69 $\pm$ 0.11 & TLE & 2.52 $\pm$ 0.1 & TLE & TLE \\ \hline
Deep Q-Learning Strategy        & 1.92 $\pm$ 0.09 & 2.93 $\pm$ 0.09 & 6.2 $\pm$ 0.11 & 2.42 $\pm$ 0.1 & 4.88 $\pm$ 0.12 & 22.2 $\pm$ 0.14 & 2.5 $\pm$ 0.11 & 6.43 $\pm$ 0.11 & 53.49 $\pm$ 0.25 \\ \hline
Threshold Strategy              & 2.18 $\pm$ 0.13 & 3.32 $\pm$ 0.13 & 6.44 $\pm$ 0.13 & 2.58 $\pm$ 0.12 & 5.58 $\pm$ 0.17 & 24.12 $\pm$ 0.19 & 2.99 $\pm$ 0.12 & 7.02 $\pm$ 0.13 & 56.73 $\pm$ 0.29 \\ \hline 

\end{tabular}}
\end{table}


We can see that the Always Accept and Random Strategies are consistently poorly performing. The DP Strategy is not shown to be exactly the best for reasons outlined above. The Threshold Strategy, even though asymptotically shown to be optimal is outperformed by several other strategies. The difference is even more significant for larger $m$, when the constant threshold (on the primary allocations) is too much of a restriction. The Mean Thinning, Local Reward Optimiser and Deep Q-Learning Strategies seem to be comparable to each other. I have calculated the one-sided p-values of a two sample t-test to see if the Deep Q-Learning Strategy is significantly better than the Mean Thinning Strategy for larger values of $n$ and $m$. The difference is significant for most of the cases (the p-values are $\sim 10^{-3}$ for $n=20$, $m=400$, $\sim 10^{-17}$ for $n=50$, $m=50$, $\sim 10^{-5}$ for $n=50$, $m=200$) but more data would be required for $n=50$, $m=2500$. \NOTE{A}{Double check that I am not writing something stupid. In particular, the test I applied assumes that the maximum loads are from a normal distribution and the two distributions have the same variance and none of them are actually true, at least not exactly... How to remedy this?} \NOTE{A}{Should I do some more p-values or this is enough to convince them about the professionalism of this dissertation?}

\NOTE{A}{Should I say some more, clever bullshit about the table, some general patterns, etc.? Maybe a bit more about the difference between the columns, not between the rows.}


One of the main conclusions and main surprise of the comparison is that the Local Reward Optimiser Strategy (which rejects only the maximum loads) has similar performance as the Deep Q-Learning Strategy, which was trained to optimise cumulative rewards. This demonstrates that the RL problem underlying \textsc{Two-Thinning} is a tough one, in particular, learning long-term consequences of decisions is difficult, because the impact of a single ball is very subtle. In general, making a small number of bad decisions might not even impact the final maximum load. \NOTE{A}{TODO: elaborate some more? This is a crucial point to make.}


\NOTE{A}{Write a bit about the running time of both training and evaluation. DQN would be a bit slower latency-wise but it is bearable. Note that other strategies require no training.}



\NOTE{A}{Add plot showing that it doesn't work very well for larger $n$ and $m$. Add general plots for huge $n$ and $m$. Also I can include here the tendencies for the theoretical results, e.g. plotting a logarithmic curve. Mention the difficulty in finding the constant factor of those theoretical algorithms.}



Analysing the optimal strategies for \textsc{Two-Thinning}, I formulated the following lemma.

\begin{lemma}\label{lemma: two-thinning-increasing-threshold}
The chosen thresholds of an optimal protocol are non-decreasing during an execution. That is, if for $i$th ball the protocol chooses a threshold $x$, then for every $j$th ball of the same game such that $j>i$, a threshold $y\geq x$ is chosen.
\end{lemma}


\NOTE{A}{Add figure of increasing threshold of DP strategy.}


This is a very surprising lemma, because the theoretically asymptotically optimal protocols in the literature don't have this property. We think that this is because they are optimal only up to a constant (or logarithmic) factor, and they are easier to reason about analytically and prove bounds on them (usually these protocols consist of independent phases with different behaviour).


\begin{proof}

\NOTE{A}{Do proper proof.}

The lemma has been verified by the optimal strategy(s) generated by dynamic programming, for several combinations of $n$ and $m$. 
\end{proof}



\NOTE{A}{Show different objective functions as well.}

\subsection{Deep Q-Learning analysis}



\subsubsection{Threshold ``Spikes'' During Training and During Execution}



\subsubsection{Training}

\NOTE{A}{There is no overfitting as there is no data, but still performance doesn't improve much after some time.}

\NOTE{A}{Explain why after a few epochs the score start to get worse for some time. But also, the strategy after 0 or 1 epochs is pretty good, as it is still a threshold based strategy, and basically random so balances out high and low thresholds.}



\subsubsection{Hyperparameter Analysis}

\NOTE{A}{Optimal max threshold can be based on theoretical bounds and baseline models (e.g. not worth using a threshold over our target result)}

\section{K-Thinning}



\subsection{Comparison of Strategies}



\begin{table}[h!]
\caption{Average maximum load of \textsc{K-Thinning} strategies with $95\%$ confidence intervals}
\label{tab:k-thinning-comparison}
\centering
\resizebox{\textwidth}{!}{%
\begin{tabular}{|l|c|c|c|c|c|c|c|c|}
\hline
                                & \multicolumn{4}{c|}{$n=5$} & \multicolumn{4}{c|}{$n=25$}\\ \hline
                                & \multicolumn{4}{c|}{$m=20$} & \multicolumn{4}{c|}{$m=50$}\\ \hline
                                & $k=2$ & $k=3$ & $k=5$ & $k=10$ & $k=2$ & $k=3$ & $k=5$ & $k=10$ \\ \hline
Always Accept Strategy          & 7.88 $\pm$ 0.14 & 7.72 $\pm$ 0.11 & 7.81 $\pm$ 0.14 & 7.8 $\pm$ 0.14 & 5.65 $\pm$ 0.19 & 5.81 $\pm$ 0.13 & 5.84 $\pm$ 0.14 & 6.03 $\pm$ 0.16 \\ \hline
Random Strategy                 & 7.66 $\pm$ 0.14 & 7.76 $\pm$ 0.13 & 7.77 $\pm$ 0.14 & 7.74 $\pm$ 0.14 & 5.77 $\pm$ 0.19 & 5.96 $\pm$ 0.15 & 5.87 $\pm$ 0.15 & 5.72 $\pm$ 0.14 \\ \hline
Local Reward Optimiser Strategy & 6.08 $\pm$ 0.06 & 5.77 $\pm$ 0.05 & 5.45 $\pm$ 0.06 & 5.13 $\pm$ 0.04 & 4.22 $\pm$ 0.11 & 3.69 $\pm$ 0.06 & 3.21 $\pm$ 0.06 & 3.0 $\pm$ 0.01 \\ \hline
Mean Thinning Strategy          & 6.17 $\pm$ 0.09 & 5.96 $\pm$ 0.04 & 5.61 $\pm$ 0.06 & 5.15 $\pm$ 0.04 & 4.34 $\pm$ 0.12 & 3.69 $\pm$ 0.07 & 3.15 $\pm$ 0.05 & 3.0 $\pm$ 0.01 \\ \hline
DP Strategy & 6.06 $\pm$ 0.1   & 5.75 $\pm$ 0.05 & 5.38 $\pm$ 0.05 & 5.1 $\pm$ 0.03 & 4.13 $\pm$ 0.09 & 3.58 $\pm$ 0.07 & 3.15 $\pm$ 0.05 & 3.0 $\pm$ 0.0 \\ \hline Threshold Strategy              & 6.44 $\pm$ 0.13 & 6.27 $\pm$ 0.05 & 6.12 $\pm$ 0.05 & 5.93 $\pm$ 0.05 & 4.72 $\pm$ 0.15 & 4.17 $\pm$ 0.05 & 4.0 $\pm$ 0.05 & 3.94 $\pm$ 0.05 \\ \hline
Deep Q-Learning Strategy        & 6.2 $\pm$ 0.11 & 6.77 $\pm$ 0.11 & 7.14 $\pm$ 0.14 & 7.14 $\pm$ 0.14 & 4.26 $\pm$ 0.09 & 4.47 $\pm$ 0.09 & 4.21 $\pm$ 0.08 & 3.01 $\pm$ 0.01 \\ \hline 

\end{tabular}}
\end{table}


\NOTE{A}{Analyse value of K and its impact.}

\begin{lemma} \label{lemma: k-thinning-increasing-threshold}
In an optimal strategy, it is not possible that it would accept a bin with $x$ choices left, but reject the same bin (with the same loads) with $y<x$ choices left. In other words, an optimal strategy should never become more selective after rejecting some options.
\end{lemma}


\begin{proof}
Note that Lemma \ref{lemma: thresholdproperty} generalises to \textsc{K-Thinning} as well - an optimal strategy must base its decisions on a threshold.


\NOTE{A}{TODO: add real proof}.


Using the dynamic programming algorithm, that creates the optimal strategy(s), I verified the lemma, that is that the threshold is indeed non-decreasing within the same load configuration.
\end{proof}


\subsection{Deep Q-Learning analysis}


\subsubsection{Training}


\subsubsection{Hyperparameter Analysis}


\NOTE{A}{Compare zero potential with max load potential, showing P value that we do need reward shaping.}


\section{Graphical Two-Choice}


\subsection{Comparison of Strategies}


\begin{table}[h!]
\caption{Average maximum load of \textsc{Graphical Two Choice} strategies with $95\%$ confidence intervals}
\label{tab:graphical-two-choice-comparison}
\centering
\resizebox{\textwidth}{!}{%
\begin{tabular}{|l|c|c|c|c|c|c|c|c|c|}
\hline
                                & \multicolumn{3}{c|}{$n=4$} & \multicolumn{3}{c|}{$n=16$} & \multicolumn{3}{c|}{$n=32$}\\ \hline
                                & \multicolumn{3}{c|}{$m=25$} & \multicolumn{3}{c|}{$m=50$} & \multicolumn{3}{c|}{$m=32$}\\ \hline
                                & Cycle & Clique & Hypercube & Cycle & Clique & Hypercube & Cycle & Clique & Hypercube \\ \hline
Greedy Strategy & 7.04 $\pm$ 0.03 & 7.04 $\pm$ 0.03 & 7.17 $\pm$ 0.06 & 4.74 $\pm$ 0.08 & 4.47 $\pm$ 0.11 & 4.39 $\pm$ 0.07 & 2.46 $\pm$ 0.1 & 2.34 $\pm$ 0.09 & 2.26 $\pm$ 0.06 \\ \hline
Random Strategy  & 8.92 $\pm$ 0.17 & 8.81 $\pm$ 0.18 & 8.98 $\pm$ 0.17 & 6.75 $\pm$ 0.17 & 6.64 $\pm$ 0.23 & 6.56 $\pm$ 0.15 & 3.45 $\pm$ 0.13 & 3.64 $\pm$ 0.16 & 3.54 $\pm$ 0.1 \\ \hline
Local Reward Optimiser Strategy & 7.1 $\pm$ 0.04 & 7.12 $\pm$ 0.05 & 7.29 $\pm$ 0.07 & 4.97 $\pm$ 0.1 & 4.74 $\pm$ 0.12 & 4.75 $\pm$ 0.07 & 2.53 $\pm$ 0.11 & 2.42 $\pm$ 0.1 & 2.46 $\pm$ 0.07 \\ \hline
DP Strategy & 7.01 $\pm$ 0.02 & 7.01 $\pm$ 0.02 & 7.27 $\pm$ 0.1 & TLE & TLE & TLE & TLE & TLE & TLE \\ \hline
Deep Q-Learning Strategy & 7.17 $\pm$ 0.07 & 7.17 $\pm$ 0.06 & 7.38 $\pm$ 0.09 & 4.92 $\pm$ 0.15 & 5.71 $\pm$ 0.17 & 6.57 $\pm$ 0.15 & 3.81 $\pm$ 0.18 & 3.59 $\pm$ 0.18 & 4.17 $\pm$ 0.14 \\ \hline 

\end{tabular}}
\end{table}


\NOTE{A}{Does connectedness matter for small graphs? Add analysis for random regular graphs.}


\begin{lemma} \label{lemma: greedy-suboptimal}
There exist a graph, such that the Greedy strategy is suboptimal with respect to the expected final maximum load of \textsc{Graphical Two Choice}.
\end{lemma}

\begin{proof}
It is enough to show that there is a state $s$ (i.e. a load vector $v$, edge $e$) where is is better to choose the bin with the larger load. This suffices because there is a non-zero probability of reaching $s$, and hence a strategy which agree with Greedy everywhere else except for this state has a strictly better expected score.


I used the dynamic programming algorithm to find the optimal strategy for \textsc{Graphical Two Choice}, and then I searched the strategy for a state where it disagrees with Greedy. I found a very small counterexample for the Cycle graph with $n=4$ bins $m=6$ balls. Denoting the nodes as ($1$-based) indices in the load vector, the counterexample state is $l=(2,0,1,0)$, $e=(2,3)$, i.e. there is an edge between the second and third bin. The reason why it is better to choose bin $3$ even though it has larger load than bin $2$ is that after that, from load vector $(2,0,2,0)$ with $2$ balls remaining we can definitely avoid having a maximum load $3$ by always placing the ball in an even-indexed bin. On the other hand, from load vector $(2,1,1,0)$, if both of the remaining edges are $(1,2)$, then there will be a maximum load of at least $3$ using whatever strategy. Therefore, the expected final maximum load of choosing bin $3$ is $2.0$, while that of bin $2$ is between $2.0$ and $3.0$.
\end{proof}

\subsection{Deep Q-Learning analysis}


\subsubsection{Training}


\subsubsection{Hyperparameter Analysis}



\section{Further Analysis of Two-Thinning (\textit{Extension})}


\subsection{Distribution of the States for an Optimal Strategy}

\NOTE{A}{we can observe an exponentially decreasing function!}
\NOTE{A}{Entropy is fairly low - this could be used as a speedup for RL training (or even as the curriculum for pretraining)}


\subsection{Explaining Trained Models}