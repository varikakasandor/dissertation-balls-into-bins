%!TEX root = ../thesis.tex
\chapter{Stabilising Training}\label{stabilising-training} 



\paragraph{Experience Replay}

Experience replay~\cite{lin1992experiencereplay} aims to address the problem that in vanilla \DQL, subsequent update steps are correlated but theoretical guarantees hold only for i.i.d. training samples -- e.g.\  in our case, after we update the weights of the DQN around a load vector with $x$ balls, we next update it around a load vector with $x+1$ balls. This correlation stems from updating in the same order as executing the actions in the game. With experience replay, instead of calling the update rule~\eqref{eq:deep-q-learning-update-with-semi-gradient} on the current $(s, a, s', r)$ tuple, we store this tuple in an experience replay buffer. Then, after every fixed number of steps, a batch of tuples are sampled uniformly at random from the buffer, and the DQN is updated according to those tuples. Another benefit is the reuse of samples, leading to a more efficient learning. Nevertheless, there should be a size limit on the buffer to get rid of outdated samples, which I implemented using a deque data structure with a limited memory capacity. Finally, note that sampling a batch of tuples rather than a single sample when updating has been shown to have a stabilising effect~\cite{qian2020batchingsgd}.


\paragraph{Target Network}


The main difference between Q-Learning and \DQL is that in the former, during one step we only update a single value in the Q-table, while in the latter, due to the function approximation, many neighbouring values are also affected. This becomes problematic, since looking at the update rule~\eqref{eq:deep-q-learning-update-with-semi-gradient} for \DQL, to update around a Q-value $Q_{\mathbf{w_t}}(s_t, a_t)$, a neighbouring Q-value $Q_{\mathbf{w_t}}(s_{t+1}, a')$ is used. This can lead to a blow-up in the Q-values, intuitively, as a chain reaction. Target networks~\cite{fan2020target} have two copies of the same network architecture: one whose weights are updated by the update rule, and one that provides $Q_{\mathbf{w_t}}(s_{t+1}, a')$ of the ``target'' part $r_{t+1}+ \max_{a'} Q_{\mathbf{w'_t}}(s_{t+1}, a')$ of the rule. This weights $w'$ of the target network are updated to that of the main network periodically, every fixed number of training episodes. Another way to think about this is as a way to achieve the stability of~\eqref{eq:deep-q-learning-update-with-semi-gradient}, while still using the full gradient descent update rule.


\paragraph{Using a Different Optimiser}


Expanding on the update rule~\eqref{eq:deep-q-learning-update-with-semi-gradient}, which updates the weights (based on the gradient) using simple Stochastic Gradient Descent (SGD), SGD could be replaced by more sophisticated optimisers taking into moving averages and variances of the gradient (e.g.\ ADAM~\cite{kingma2015adamoptimiser}. I also added the widely used gradient clipping \cite{mikolov2012gradientclippingoriginal} technique which does not allow absolute gradient values to exceed $1$, hence prevents exploding gradients and speeding up training~\cite{zhang2020gradientclippingaccelerate}.

