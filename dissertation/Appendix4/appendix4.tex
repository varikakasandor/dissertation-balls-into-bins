%!TEX root = ../thesis.tex
\chapter{\KThinning Implementation Details}\label{k-thinning details} 


\section*{\DQL Implementation}


To formulate the \KThinning problem as a MDP, I decided to use (load vector, number of choices left) pairs as states, and just like for \TwoThinning, thresholds as actions. The ``number of choices left'' part of the states indicate how many more bins can be rejected before the ball would be allocated into a randomly chosen bin. The state transitions happen according to the definition of \KThinning. To tackle the sparse reward problem, I decided to use the same potential functions as for \TwoThinning (hence not taking into account the number of choices left) to avoid introducing unnecessary bias for the agent.


For the DQN, I extended the architecture for \TwoThinning by concatenating the final hidden state of the RNN with the one-hot encoded representation of the number of choices left, feeding them together to the linear layers. The rest of the implementation is analogous to \TwoThinning.



\section*{Dynamic Programming}

For \KThinning, I implemented a DP algorithm analogous to that for \TwoThinning, with the states being (sorted load vector, number of choices left) pairs in this case.


The recurrence equations are:

\begin{align} 
    V_{\pi^*}((v, 0)) &= \max_a \mathbb{E} [r_{t+1} + V_{\pi^*}(s_{t+1}) \mid s_t=(v,0), a_t=a] \notag \\
    &= \max_{i \in [n]} \mathbb{E} [r_{t+1} + V_{\pi^*}(s_{t+1}) \mid s_t=(v,0), a_t=v[i]] \notag \\
    &= \max_{i \in [n]} \left(\sum_{0\leq j \leq i} \frac{1}{n}\cdot V_{\pi^*}((v+e_j,k)) + \frac{n-i-1}{n} \cdot  \sum_{j \in [n]} \frac{1}{n}\cdot V_{\pi^*}((v+e_j,k))\right) \text{ ,} \label{eq:kthinning-dynamicprogramming} \\
    V_{\pi^*}((v, x+1)) &= \max_a \mathbb{E} [r_{t+1} + V_{\pi^*}(s_{t+1}) \mid s_t=(v,x+1), a_t=a] \notag \\
    &= \max_{i \in [n]} \mathbb{E} [r_{t+1} + V_{\pi^*}(s_{t+1}) \mid s_t=(v, x+1), a_t=v[i]] \notag \\
    &= \max_{i \in [n]} \left(\sum_{0\leq j \leq i} \frac{1}{n}\cdot V_{\pi^*}((v+e_j,k)) + \frac{n-i-1}{n} \cdot  V_{\pi^*}((v, x))\right) \text{ ,}
\end{align}

and the base case is $V_{\pi^*}((v, k))=-\mathrm{maxload}(v)$ for $\mathrm{sum}(v)=m$. The runtime of this algorithm is $k$ times the runtime of the DP algorithm for \TwoThinning: $\Theta(|S|\cdot n\cdot k)$.